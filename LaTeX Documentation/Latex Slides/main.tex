\documentclass{beamer}
\usepackage{setspace}
\usepackage{polyglossia} 
\usepackage{hyperref}
\usetheme{Singapore}
\usepackage{xcolor} 
\setdefaultlanguage{english}  
\setotherlanguage{hindi} 
\newfontfamily\devanagarifont[Script=Devanagari]{Lohit Devanagari} 
\newfontfamily\devanagarifontsf[Script=Devanagari]{Lohit Devanagari}
\usepackage{graphicx} % Required for inserting images
\title{\textbf{\large{Role of Impathy and Integrity in Professional life}}}
\author{
\large{By} \\
\itemsep
\large{\textbf{Tushar Dhakad[2023ucp1977]}}\\}
\institute{
\includegraphics[width=0.1\textwidth]{pictures/Logo.png}\\
\small{\texthindi{मालवीय राष्ट्रीय प्रौद्योगिकी संस्थान जयपुर}}\\
\small{\textbf{Malviya National Institute of Technology Jaipur}}\\
\tiny{\textbf{[AN INSTITUTE OF NATIONAL IMPORTANCE]}}\\
\includegraphics[width=1\textwidth]{pictures/sketch.png}
}
\date{}
\usetheme{Singapore}
\begin{document}
\begin{frame}[plain]
\titlepage
\end{frame}
\section{Section 1}
\subsection{sub a}
\begin{frame}
\frametitle{Empathy}
\large{Empathizing with others is essential for healthy\\
relationships and communication. After all, it's hard to know \\how to relate to others if you can't understand their feelings.\\
Empathy is the ability to emotionally understand what other people feel,\\ see things from their point of view, and imagine\\ yourself in their place. Essentially, it is putting yourself in
\\someone else's position and feeling what they are feeling.}
\end{frame}
\begin{frame}
    \centering
    \textbf{\huge{Thank You}}
\end{frame}

\end{document}

